\documentclass[a4paper,11pt,openright]{memoir}


% Pakker indsættes herefter:
\usepackage[danish]{babel}
\usepackage[utf8x]{inputenc}
\usepackage[footnote,draft,danish,silent,nomargin]{fixme}       % draft istedet for final, gør det muligt at compile med fixmes

\usepackage{hyperref}						% Laver indholdsfortegnelse, referencer og bibliografi til hyperlinks
\usepackage[T1]{fontenc}
\usepackage{graphicx}						% Indsættelse af billeder mm.
\usepackage{amsmath,amssymb,mathtools,bm}	% Diverse matematik
\usepackage{color,xcolor}
\usepackage{pdfpages}						% Giver mulighed for at sætte hele pdf sider ind f.eks. en forside
\usepackage{natbib}							% Til referencer hører sammen med \bibliographystyle
\usepackage{listings}						% Præsentering af kode
\usepackage{booktabs}
\usepackage{rotating}

% Setup af dokumentet:
\chapterstyle{ell}
\setsecnumdepth{subsection}
\settocdepth{subsection}
\pretolerance=2500							% Sætter tolerancen for orddeling; jo højere, jo mindre orddeling
\bibliographystyle{plainnat}				% Hvordan referencer vises i teksten

\setlength{\parindent}{1.5mm}				% Størrelse af indryk
\setlength{\parskip}{1.8mm}					% Afstand mellem afsnit ved brug af "double Enter"
% \linespread{1,1}							% Linie afstand

\pagestyle{ruled}							% Header og Footer - "plain" for intet, ud over sidetal i bunden


\setlrmarginsandblock{3.5cm}{2.5cm}{*}		% \setlrmarginsandblock{Indbinding}{Kant}{Ratio}
\setulmarginsandblock{2.5cm}{3.0cm}{*}		% \setulmarginsandblock{Top}{Bund}{Ratio}
\checkandfixthelayout						% Laver forskellige beregninger og sætter de almindelige længder op

\definecolor{dkgreen}{rgb}{0,0.6,0}
\definecolor{gray}{rgb}{0.5,0.5,0.5}
\definecolor{mauve}{rgb}{0.58,0,0.82}

\lstdefinelanguage{pbasm}
{
morekeywords={JUMP,CALL,RETURN,ADD,ADDCY,SUB,SUBCY,COMPARE,RETURNI,ENABLE,
RETURNI,DISABLE,ENABLE,INTERRUPT,DISABLE,INTERRUPT,LOAD,OR,AND,XOR,
TEST,STORE,FETCH,SR0,SR1,SRX,SRA,RR,SL0,SL1,SLX,SLA,RL,INPUT,OUTPUT,CONSTANT,NAMEREG},
sensitive=false,
morecomment=[l]{;},
alsoletter={\$}
}

\lstset{frame=tb,
  language=C,
  aboveskip=3mm,
  belowskip=3mm,
  showstringspaces=false,
  columns=flexible,
  basicstyle={\small\ttfamily},
  stepnumber=2,
  firstnumber=1,
  escapeinside={@}{@},
  numberfirstline=false,
  numbers=left,
  numbersep=8pt,
  numberstyle=\tiny\color{gray},
  keywordstyle=\color{blue},
  commentstyle=\color{dkgreen},
  stringstyle=\color{mauve},
  breaklines=true,
  breakatwhitespace=true
  tabsize=3,
  captionpos=b,
  keepspaces=true
}

% Makroer:
\renewcommand\appendixtocname{Bilag}
\newcommand{\ohm}{\ensuremath{\mathrm{\Omega}}}
\newcommand{\n}{\hfill\newline}					% Laver ny linje unden warnings
