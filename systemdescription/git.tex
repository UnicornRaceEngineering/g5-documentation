\section{Git}

To get a copy of the repositories open a git terminal and and `cd' to the
location you want your repository copies to be stored. Then you run the
command:

\begin{verbatim}
	git clone "path"
\end{verbatim}

where ``path'' is the path to the git repository on the AFS server. Remember to
use `/' (standard in Unix) instead of `\\' (standard in Windows). After cloning
you should now have a new folder called by the same name as the folder you are
cloning from on AFS without the .git ending.\n
In this project we use two repositories, ``report'' and ``code''.\n
Make sure to clone both repositories.

The you `cd' into each of the cloned repositories on your computer and run the
two commands:

\begin{verbatim}
	git config --global user.name "John Doe"
	git config --global user.email johndoe@example.com
\end{verbatim}

Where you instead of John Doe put in your own name and put in your own email
instead of the example.\n
This will insure that other people can see your name and contact details for
everything you commit.

Every time you want to do something in your repository you need to `cd' into
that folder first.\n
Here is a list of some common commands that you will most likely use at some
point:


\begin{verbatim}
1.
	git status
\end{verbatim}

This command will show you a list of changes in the directory which have yet
to commit. files shown in green are already added to the next commit and which
means they will be commit when running the command git commit.\n
Files shown in red has not yet been added.

\begin{verbatim}
2.
	git add file1 file2 ...
\end{verbatim}

This will add the files you write here to the next commit (see 1.).

\begin{verbatim}
3.
	git pull
\end{verbatim}

Get the latest changes from the repository on AAU's server. Make sure to
commit your own changes before running this command.

\begin{verbatim}
4.
	git commit -m ``message''
\end{verbatim}

Commits all the changes you have added. insert your own message to tell people
about the changes you have made. So that if you fuck up we can easily find
your commit and save the project :)\n
If you don't write a short descriptive message of what you have done - I know
where you live!

\begin{verbatim}
5.
	git commit -am ``message''
\end{verbatim}

Does the same as 4. except it automatically adds all the modified files you
have. Notice that this command DOES NOT add newly created files. Those you
will still have to add with 2.

\begin{verbatim}
6.
	git push origin master
\end{verbatim}

Push your changes to the repository on the server. That means uploading the
commits you have made (Can be multiple commits) to the server. Notice that this
means that you need to commit your changes before you can push/upload them.

\begin{verbatim}
7.
	git reset --hard origin
\end{verbatim}

This command resets your repository to that on the server. Only use this if
you fucked up and are willing to disregard any changes or commits you have in
your local repository.

\begin{verbatim}
8.
	git log
\end{verbatim}

This prints out the log in the terminal where you can see the commit messages
of all earlier commits up to your last pull or commit.
